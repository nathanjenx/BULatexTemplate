%!TEX root = ../main.tex
\chapter{Design and Implementation}
You should always start with an overview (Heading 2 style) to tell what this chapter is about and finish with a summary (Heading 2 style) to tell what has been covered in this chapter.

The Design and Implementation chapter should explain the design technique chosen and justify why it is appropriate, depending on the development methodology.  Suitable diagram-techniques (e.g. UML, other drawings) should be used where appropriate. For the Implementation part, it should talk about the technical realisation of the concepts and ideas developed earlier. It is used to describe the system at a finer level of technical details, down to the code level. However, do not attempt to describe all the code in the system, and do not include large pieces of code in this section. 

You should highlight the pieces of code which are critical to the system or worth to be noted. For example, the creation and/or implementation of core algorithms that make the system functional or some methods/ways you have used which are non-standard or innovative in the system implementation. You should also mention any unforeseen problems you encountered when implementing the system and how and to what extend you overcame them.

Appropriate testing must also be included in this section
