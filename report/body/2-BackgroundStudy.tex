%!TEX root = ../main.tex

\chapter{Background Study}
\label{chap:bgStudy}

\section{Template Text}
Figures must be correctly numbered with captions and paragraph text should not be wrapped around figures - same rules apply to tables. An example of figures can be found below.
\begin{figure}[t]
	\centering
	\includegraphics[width=0.45\textwidth]{unilogo.jpg}
	\caption{Bournemouth University}
	\label{fig:BULogo2}
\end{figure}
You should always start with an overview (Heading 2 style) to tell what this chapter is about and finish with a summary (Heading 2 style) to tell what has been covered in this chapter.

The Background Study (Research) or State of Art chapter is to provide your readers with information that they cannot be expected to know in detail but which they will need to know in order to fully understand and appreciate the rest of the dissertation. In short, it describes the research you have done in order to prepare for the project. You should use this section to demonstrate how much you really understand the problem domain in terms of previous (related) literature and existing solutions. For example, if your project is about developing a bespoke online CRM system for a client, this section is expected to answer the following questions: 
\begin{enumerate}
	\item What is CRM (Customer Relationship Management)?
	\item What are the characteristics of CRM?
	\item What are the main types/models of CRM?
	\item How CRM is usually implemented and what should be considered in the implementation?
	\item Are there any existing solutions and what are their advantages and limitations?
\end{enumerate}

\subsection{Literature Review}
Note Literature Review is not often used as the title of this chapter even if your project is research based



